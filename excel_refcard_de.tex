\documentclass[8pt]{extarticle} % extarticle: font sizes < 10

\usepackage[
      pdftitle={Excel Shortcuts for the highly Effcient},
      pdfauthor={mjt91},
      pdfkeywords={Excel, Quick Reference, Refcard, Cheat Sheet},
      pdfsubject={Reference Card for Excel Shortcuts}
]{hyperref}

\usepackage{refcards}

\usepackage{vmargin}
% A4
\setpapersize[portrait]{A4}
\setmarginsrb%
{1.5cm}  % left
{1.0cm}  % top
{1.5cm}  % right
{1.0cm}  % bottom
{0ex}    % header height
{0ex}    % header separation
{0ex}    % footer height
{0ex}    % footser separation
\setlength\columnsep{7mm}

% Letter
% \setpapersize[landscape]{USletter}
% \setmarginsrb%
% {1.1cm}  % left
% {1.1cm}  % top
% {0.9cm}  % right
% {0.9cm}  % bottom
% {0ex}    % header height
% {0ex}    % header separation
% {0ex}    % footer height
% {0ex}    % footser separation
% \setlength\columnsep{4mm}

\begin{document}
\raggedright

\begin{multicols}{2}

\title{Excel Kurzbefehle für die Hocheffizienten}

{\small
(c) 2024 mjt91 \url{<mjt91@omg.lol>}\\
\url{http://www.mjt91.url.lol/blog}

This work is licensed under the Creative Commons Attribution-Noncommercial-Share
Alike 3.0 License. To view a copy of this license, visit
\url{http://creativecommons.org/licenses/by-nc-sa/}
}

\vspace*{1pt}

\section{Allgemein}

  \vspace{1ex}
%   \subsection{Simple Data Types}
  \begin{tabular}{L{0.55\linewidth} L{0.45\linewidth}}

    \tt ALT + F9                & \"Offne das VBA-Tools-Fenster \\
    \tt CTRL + N                & Neues Arbeitsblatt öffnen \\
    \tt CTRL + P                & Druckvorschau öffnen \\
    \tt CTRL + ;                & Aktuelles Datum einfügen \\
    \tt CTRL + L                & Tabelle erstellen \\
    \tt CTRL + K                & Hyperlink einfügen \\
    \tt CTRL + D                & Flash-Fill von oben \\
    \tt CTRL + +                & Neue Spalte/Zeile einfügen, falls ganze Spalte/Zeile ausgewählt ist \\
    \tt CTRL + F                & Suchdialog öffnen \\
    \tt CTRL + H                & Suchen \& Ersetzen-Dialog öffnen \\

  \end{tabular}
  
\section{Navigation}

  \begin{tabular}{L{0.55\linewidth} L{0.45\linewidth}}

    \tt CTRL + SPACE            & Ganze Spalte auswählen \\
    \tt SHIFT + SPACE           & Ganze Zeile auswählen \\
    \tt SHIFT + ↓               & Zellauswahl erweitern (jeder Pfeiltaste funktioniert) \\
    \tt SHIFT + CTRL + ↓        & Alle Zellen bis zum Ende/Anfang nach unten/links/rechts/oben auswählen (jeder Pfeiltaste funktioniert) \\
    \tt CTRL + ↓                & Zum letzten/ersten Zelle in der aktuell ausgewählten Spalte/Zeile springen (jeder Pfeiltaste funktioniert) \\

  \end{tabular}


\section{Formatierung}

  \begin{tabular}{L{0.55\linewidth} L{0.45\linewidth}}
    
    \tt CTRL + SHIFT + B        & \textbf{Fett} machen \\
    \tt CTRL + SHIFT + I        & \textit{Kursiv} machen \\
    \tt CTRL + SHIFT + U        & \underline{Unterstreichen} \\
    \tt CTRL + 1                & Fenster für Formatierungswerkzeuge öffnen \\
    \tt CTRL + SHIFT + \#       & Datumsformat anwenden \\
    \tt CTRL + SHIFT + \&       & Rahmen zur Auswahl hinzufügen \\
    
  \end{tabular}


\end{multicols}
\end{document}
